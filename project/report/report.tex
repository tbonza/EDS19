\documentclass{article}

% if you need to pass options to natbib, use, e.g.:
%     \PassOptionsToPackage{numbers, compress}{natbib}
% before loading neurips_2018

% ready for submission
% \usepackage{neurips_2018}

% to compile a preprint version, e.g., for submission to arXiv, add add the
% [preprint] option:
%     \usepackage[preprint]{neurips_2018}

% to compile a camera-ready version, add the [final] option, e.g.:
\usepackage[final]{nips_2018}

% to avoid loading the natbib package, add option nonatbib:
%     \usepackage[nonatbib]{neurips_2018}

\usepackage[utf8]{inputenc} % allow utf-8 input
\usepackage[T1]{fontenc}    % use 8-bit T1 fonts
\usepackage{hyperref}       % hyperlinks
\usepackage{url}            % simple URL typesetting
\usepackage{booktabs}       % professional-quality tables
\usepackage{amsfonts}       % blackboard math symbols
\usepackage{nicefrac}       % compact symbols for 1/2, etc.
\usepackage{microtype}      % microtypography
\usepackage{graphicx}
\usepackage{subcaption}
\usepackage{amsmath}
\usepackage{multirow} 


\graphicspath{ {imgs/} }

\title{Repo Velocity: Diagnosing Git Repository Health}

% The \author macro works with any number of authors. There are two commands
% used to separate the names and addresses of multiple authors: \And and \AND.
%
% Using \And between authors leaves it to LaTeX to determine where to break the
% lines. Using \AND forces a line break at that point. So, if LaTeX puts 3 of 4
% authors names on the first line, and the last on the second line, try using
% \AND instead of \And before the third author name.

\author{%
  Tyler Brown\\
  \texttt{NUID: 001684955} \\
  % Coauthor \\
  % Affiliation \\
  % Address \\
  % \texttt{email} \\
  % \AND
  % Coauthor \\
  % Affiliation \\
  % Address \\
  % \texttt{email} \\
  % \And
  % Coauthor \\
  % Affiliation \\
  % Address \\
  % \texttt{email} \\
  % \And
  % Coauthor \\
  % Affiliation \\
  % Address \\
  % \texttt{email} \\
}

\begin{document}
% \nipsfinalcopy is no longer used

\maketitle

\section{Introduction}

Understanding the health of open source projects is important to
industry because it helps them assess risk associated with their
technology stack. Forecasting health is important to investors because
this information can help them make profitable investments in open
source. Academia is interested to know if there are links between
theory, such as programming language design \cite{ray2014large}, and the
health of open source software. A previous metric used to assess open source
software health is the Truck Factor \cite{avelino2015truck}. Truck Factor
is the smallest subset size of developers who contributed 50\% of the code
in an open source project. The underlying intuition is that open source
projects with a lower Truck Factor are more susceptible to project disruption
in the event of adverse circumstances.

By borrowing from Physics, this paper
\footnote{Code available on GitHub: https://github.com/tbonza/EDS19}
contributes to the advancement of understanding the health of open source
projects, studying a subset of GitHub repositories, with the following:

\begin{itemize}
\item Introducing a health measure that can be assessed at time
  $t_{i \pm k}$ where $k$ is a multiple of $i$.
\item New health measure can be used in forecasting as well as description
\item Health measure is rooted in Physics so we can derive related measures
  using preexisting theory.
\end{itemize}

\section{Related Work}

Truck Factor reflects robustness of project \cite{avelino2015truck} by
computing the minimum number of developers required to comprise 50\% of
file ownership. The underlying intuition is that a project with a low
number of dominant developers will be more susceptible to failing due
to external shocks. Projects where file ownership is shared by a large
number of developers are interpreted as more healthy with this approach.

Not all projects reflects software which requires support. Researchers have
classified many project types such as 'software development projects',
'solutions for homework', 'projects with educational purposes', 'data sets',
and 'personal web sites' \cite{soll2017classifyhub}. The observation that
many projects are not software development is also a noted peril when
analyzing GitHub \cite{kalliamvakou2016depth}. We may also find repositories
containing code duplicates \cite{lopes2017dejavu}. These factors may
comprise threats to validity which are important to address in this analysis.

\section{Methods}

The Truck Factor assesses Git repository health at time $t_i$. We would
like to better understand health at different points in time $t_{i \pm k}$
where $k \ge 1$. We can borrow from Physics to find such a measure. Velocity,
$v(t)$, is a measure of distance over time. What is a proxy for distance in
the context of Git repository health? Total lines of code changed in a
project during time $t_i$ can tell us how quickly the code base is
changing. We apply this definition of velocity, $v(t)$, for our new measure
Repo Velocity.

\begin{equation} \label{eq:1}
  \text{Repo Velocity } \approx v(t) = \frac{d}{t} =
  \frac{\text{lines added } - \text{ lines deleted}}{t}
  \qquad \forall i \pm k \in t | i \in \mathbb{R}, k \in \mathbb{R}_{>0}
\end{equation}

Equation \ref{eq:1} defines our target measure, Repo Velocity, and posits
that the Physics concept of 'distance' can provide insight into how a GitHub
repository is changing over time. We can compute velocity, $v(t_i)$ at time
$t_i$ where $i$ is a real number. We can also compute average velocity,
$v_{avg}(t_{i \pm k})$, at time $t_i$ to time $t_{i \pm k}$. The Repo
Velocity measure also has the advantage of allowing aggregation and
disaggregation. Truck Factor is a measure which can only be analyzed at an
aggregate level. There can be a lot of interesting variation below a
Repository level aggregation. Repo Velocity can also be used to capture
individual contributor velocity. The ability of Repo Velocity to be applied
over time as well as at aggregated and disaggregated levels make the measure
a unique contribution to the understanding of Git repository health. The
remaining methodology seeks to establish Repo Velocity as an informative
descriptive and predictive target or dependent variable.

\subsection{Establishing Baseline Repo Velocity}

In our effort to establish a new measure which can be applied at time

\subsubsection{Descriptive Baseline}

got some trends

\subsubsection{Predictive Baseline}

using a simple prediction

\subsection{Validating Repo Velocity at Scale}

Collecting from Facebook, etc.

\subsection{Data Engineering}

Wrote a Python package, Okra.

\subsubsection{Collecting Data}

used kubernetes and ``being nice''

\subsubsection{Processing Data}

using Apache spark, using ``hack/reduce''

\section{Experiment}

\subsection{Results from Baseline Repo Velocity}

\subsubsection{Results from Predictive Baseline}

\subsubsection{Results from Scaled up Prediction}

\subsection{Results from Repo Velocity at Scale}


\subsubsection{Modeling Outliers}

bad 

\section{Discussion}

had mixed results, should have better classified input GIGO

\subsection{Threats to Validity}

 yup

%\newpage
\bibliographystyle{unsrt}
\bibliography{references}

\end{document}
