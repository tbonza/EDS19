\documentclass{article}

% if you need to pass options to natbib, use, e.g.:
%     \PassOptionsToPackage{numbers, compress}{natbib}
% before loading neurips_2018

% ready for submission
% \usepackage{neurips_2018}

% to compile a preprint version, e.g., for submission to arXiv, add add the
% [preprint] option:
%     \usepackage[preprint]{neurips_2018}

% to compile a camera-ready version, add the [final] option, e.g.:
\usepackage[final]{nips_2018}

% to avoid loading the natbib package, add option nonatbib:
%     \usepackage[nonatbib]{neurips_2018}

\usepackage[utf8]{inputenc} % allow utf-8 input
\usepackage[T1]{fontenc}    % use 8-bit T1 fonts
\usepackage{hyperref}       % hyperlinks
\usepackage{url}            % simple URL typesetting
\usepackage{booktabs}       % professional-quality tables
\usepackage{amsfonts}       % blackboard math symbols
\usepackage{nicefrac}       % compact symbols for 1/2, etc.
\usepackage{microtype}      % microtypography
\usepackage{graphicx}
\usepackage{subcaption}
\usepackage{amsmath}
\usepackage{multirow} 


\graphicspath{ {imgs/} }

\title{Diagnosing Git Repository Health}

% The \author macro works with any number of authors. There are two commands
% used to separate the names and addresses of multiple authors: \And and \AND.
%
% Using \And between authors leaves it to LaTeX to determine where to break the
% lines. Using \AND forces a line break at that point. So, if LaTeX puts 3 of 4
% authors names on the first line, and the last on the second line, try using
% \AND instead of \And before the third author name.

\author{%
  Tyler Brown\\
  \texttt{NUID: 001684955} \\
  % Coauthor \\
  % Affiliation \\
  % Address \\
  % \texttt{email} \\
  % \AND
  % Coauthor \\
  % Affiliation \\
  % Address \\
  % \texttt{email} \\
  % \And
  % Coauthor \\
  % Affiliation \\
  % Address \\
  % \texttt{email} \\
  % \And
  % Coauthor \\
  % Affiliation \\
  % Address \\
  % \texttt{email} \\
}

\begin{document}
% \nipsfinalcopy is no longer used

\maketitle

\section{Introduction}

Understanding the health of open source projects is important to
industry because it helps them assess risk associated with their
technology stack. Forecasting health is important to investors because
this information can help them make profitable investments in open
source. Academia is interested to know if there are links between
theory, such as programming language design \cite{ray2014large}, and the
health of open source software. A previous metric used to assess open source
software health is the Truck Factor \cite{avelino2015truck}. Truck Factor
is the smallest subset size of developers who contributed 50\% of the code
in an open source project. The underlying intuition is that open source
projects with a lower Truck Factor are more susceptible to project disruption
in the event of adverse circumstances.

By borrowing from Physics, this paper contributes to the advancement of
understanding the health of open source projects with the following:

\begin{itemize}
\item Introducing a health measure that can be assessed at time
  $t_{i \pm k}$ where $k$ is a multiple of $i$.
\item New health measure can be used in forecasting as well as description
\item Health measure is rooted in Physics so we can derive related measures
  using preexisting theory.
\end{itemize}

\section{Related Work}

Related to truck factor

\section{Methods}

ok

 \section{Experiment}

 \section{Discussion}

 yup

%\newpage
\bibliographystyle{unsrt}
\bibliography{references}

\end{document}
